\documentclass{article}

\usepackage{amsmath}
\usepackage{listings}

\begin{document}
	Assuming \(nums1 = [a_1, a_2, a_3, a_4, \dots, a_m]\) and \(nums2 = [b_1, b_2, b_3, b_4, \dots, b_n]\), then 
	%
	\[
		nums3 =
		\left (
		 {\begin{array}{cccccccc}
		 		
				(a_1 \oplus b_1) & \oplus & (a_1 \oplus b_2) & \oplus & \cdots & \oplus & (a_1 \oplus b_n) & \oplus \\
				(a_2 \oplus b_1) & \oplus & (a_2 \oplus b_2) & \oplus & \cdots & \oplus & (a_2 \oplus b_n) & \oplus \\
				\vdots&  & \vdots &  & \ddots &  & \vdots\\
				(a_m \oplus b_1) & \oplus & (a_m \oplus b_2) & \oplus & \cdots & \oplus & (a_m \oplus b_n) & 
		\end{array} } 
		\right )
	\]
	%
	Or equivalently, 
	\[
	nums3 =
	\left (
	{\begin{array}{cccccccccccccccc}
			a_1 & \oplus & b_1 & \oplus & a_1 & \oplus & b_2 & \oplus & \cdots & \oplus & a_1 & \oplus & b_n & \oplus \\
			a_2 & \oplus & b_1 & \oplus & a_2 & \oplus & b_2 & \oplus & \cdots & \oplus & a_2 & \oplus & b_n & \oplus \\
			    & \vdots &     & \vdots &     & \vdots &     & \vdots & \ddots & \vdots &     & \vdots &     &        \\
			a_m & \oplus & b_1 & \oplus & a_m & \oplus & b_2 & \oplus & \cdots & \oplus & a_m & \oplus & b_n & 
	\end{array} } 
	\right )
	\]
	%
	If \(m\) is even, then there will be even number of \(b_1, b_2, b_3, \dots, b_n\), which will be \(0\) after applying \(\oplus\).
	If \(m\) is odd, then some even number of \(b_1, b_2, b_3, \dots, b_n\) will become \(0\) after applying \(\oplus\), and one of each \(b_1, b_2, b_3, \dots, b_n\) will remain, thus the result will be \(b_1 \oplus b_2 \oplus b_3 \oplus \dots \oplus b_n\).
	\[
	nums3 =
	\left (
	{\begin{array}{cccccccccccccccc}
			
			a_1 & \oplus & \mathbf{b_1} & \oplus & a_1 & \oplus & \mathbf{b_2} & \oplus & \cdots & \oplus & a_1 & \oplus & \mathbf{b_n} & \oplus \\
			a_2 & \oplus & \mathbf{b_1} & \oplus & a_2 & \oplus & \mathbf{b_2} & \oplus & \cdots & \oplus & a_2 & \oplus & \mathbf{b_n} & \oplus \\
			    & \vdots &              & \vdots &     & \vdots &              & \vdots & \ddots & \vdots &     & \vdots &              &        \\
			a_m & \oplus & \mathbf{b_1} & \oplus & a_m & \oplus & \mathbf{b_2} & \oplus & \cdots & \oplus & a_m & \oplus & \mathbf{b_n} & 
	\end{array} } 
	\right )
	\]
	Similarly, if \(n\) is even, there will be even number of \(a_1, a_2, a_3, \dots, a_m\), which will be \(0\) after applying \(\oplus\).
	And when \(n\) is odd, the result of applying \(\oplus\) will be \(a_1 \oplus a_2 \oplus a_3 \oplus \dots \oplus a_m\).
	\[
	nums3 =
	\left (
	{\begin{array}{cccccccccccccccc}
			\mathbf{a_1} & \oplus & b_1 & \oplus & \mathbf{a_1} & \oplus & b_2 & \oplus & \cdots & \oplus & \mathbf{a_1} & \oplus & b_n & \oplus \\
			\mathbf{a_2} & \oplus & b_1 & \oplus & \mathbf{a_2} & \oplus & b_2 & \oplus & \cdots & \oplus & \mathbf{a_2} & \oplus & b_n & \oplus \\
			             & \vdots &     & \vdots &              & \vdots &     & \vdots & \ddots & \vdots &              & \vdots &     &        \\
			\mathbf{a_m} & \oplus & b_1 & \oplus & \mathbf{a_m} & \oplus & b_2 & \oplus & \cdots & \oplus & \mathbf{a_m} & \oplus & b_n & 
	\end{array} } 
	\right )
	\]
	Hence we have our code.
\end{document}